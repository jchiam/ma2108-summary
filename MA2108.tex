\documentclass[10pt, twocolumn]{article}
\usepackage[landscape]{geometry}
\usepackage[latin1]{inputenc}
\usepackage{amsmath}
\usepackage{amsfonts}
\usepackage{amssymb}
\usepackage{multicol}
\usepackage{array}
\usepackage{setspace}
\title{MA2108 Summary}

\renewcommand{\arraystretch}{1.5}
\newcommand{\indentitem}{\setlength\itemindent{25pt}}
\setlength{\columnsep}{20pt}

\begin{document}
\begin{center}
\part*{MA2108 Summary}
\end{center}
\tableofcontents
\newpage
\onehalfspacing

\section{The Real Numbers}
\subsection{The Algebraic Properties of $\mathbb{R}$}
$\mathbb{R}$ is a complete ordered field.
\newline
$\mathbb{R}$ is a \underline{field} because it has the following algebraic properties:
\begin{enumerate}
\item[1.]{$a+b=b+a$ \hspace{20pt} $\forall a,b \in \mathbb{R}$}
\item[2.]{$(a+b)+c=a+(b+c)$ \hspace{20pt} $\forall a,b,c \in \mathbb{R}$}
\item[3.]{$\exists 0 \in \mathbb{R}$ such that $0+a=a+0=a$ \hspace{20pt} $\forall a \in \mathbb{R}$}
\item[4.]{For each $a \in \mathbb{R}, \exists -a \in \mathbb{R}$ such that}
\begin{center}
$a+(-a)=(-a)+a=0$
\end{center}
\item[5.]{$ab=ba, \forall a,b \in \mathbb{R}$}
\item[6.]{$(ab)c=a(bc), \forall a,b,c \in \mathbb{R}$}
\item[7.]{$\exists 1 \in \mathbb{R}$ such that $1 \neq 0$ and $1a=a1=a \forall\in\mathbb{R}$}
\item[8.]{If $a \in \mathbb{R}$ and $a \neq 0$, then $\exists \frac{1}{a} \in \mathbb{R}$ such that}
\begin{center}
$a \cdot \frac{1}{a} = \frac{1}{a} \cdot a = 1$
\end{center}
\item[9.]{$a(b+c)=ab+ac, \forall a,b,c \in \mathbb{R}$}
\end{enumerate}
Any nonempty set $F$ together with two binary operations called \underline{addition} and \underline{multiplication} satisfying the above conditions is called a \underline{field}.

\subsection{The Ordered Properties of $\mathbb{R}$}
{\bf The Trichotomy Property}
\newline
If $a,b \in \mathbb{R}$, then exactly one of the following holds:
\begin{center}
$a<b$ \hspace{20pt} $b<a$ \hspace{9pt} or \hspace{9pt} $a=b$
\end{center}
\begin{enumerate}
\item[(a)]{$a<b$ and $b<c \Rightarrow a<c$}
\item[(b)]{$a<b \Rightarrow a+c<b+c, \forall c \in \mathbb{R}$}
\item[(c)]{$a<b $and $c>0 \Rightarrow ac<bc$ and $a<b and c<0 \Rightarrow ac>bc$}
\end{enumerate}

\subsection{Intervals}
{\bf Open Interval}
\begin{center}
$(a,b) = {x\in\mathbb{R} : a<x<b}$
\end{center}
{\bf Closed Interval}
\begin{center}
$[a,b] = {x\in\mathbb{R} : a \leq x \leq b}$
\end{center}

\subsection{Solving Inequalities}
{\bf Rule 1:} If $ab>0$, either
\begin{enumerate}
\item[(i)]{$a>0$ and $b>0$}
\item[(ii)]{$a<0$ and $b<0$}
\end{enumerate}
{\bf Rule 2:} If $ab<0$, either
\begin{enumerate}
\item[(i)]{$a<0$ and $b>0$}
\item[(ii)]{$a>0$ and $b<0$}
\end{enumerate}
{\bf Example}
\newline
Solve $\frac{ax+b}{cx+d}<r$
\begin{enumerate}
\item[1.]{Bring $r$ to LHS and combine to single fraction.}
\item[2.]{Multiply both sides by denominator squared.}
\item[3.]{Solve quadratic inequality.}
\end{enumerate}
{\bf Bernoulli's Inequality}
\begin{center}
$(1+x)^n \geq 1+nx$ \hspace{20pt} $\forall n \in \mathbb{N}$, $x>-1$
\end{center}

\subsection{AM-GM-HM}
{\bf Definitions}
\newline
Given $a_1,a_2,\dots,a_n$
\begin{enumerate}
\item[1.]{Arithmetic Mean}
\begin{center}
$A = \frac{a_1+a_2+\cdots +a_n}{n}$
\end{center}
\item[2.]{Geometric Mean}
\begin{center}
$G = (a_1a_2 \cdots a_n)^{\frac{1}{n}}$
\end{center}
\item[3.]{Harmonic Mean}
\begin{center}
$H = \frac{n}{\frac{1}{a_1}+\frac{1}{a_2}+ \cdots +\frac{1}{a_n}}$
\end{center}
\end{enumerate}
{\bf AM-GM-HM Inequality}
\newline
Given $a_1=a_2= \cdots = a_n$ and letting $A$, $G$, $H$ be the different means respectively, Then
\begin{center}
$H \leq G \leq A$
\end{center}

\subsection{Absolute Value}
{\bf Definition}
\newline
Let $a \in \mathbb{R}$.
\begin{center}
$|a| = 
\left\{ \begin{array}{ll}
a & \text{if} a>0 \\
-a & \text{if} a<0 \\
0 & \text{if} a=0
\end{array} \right.$
\end{center}

{\bf Properties of Absolute Value}
\begin{enumerate}
\item[1.]{test}


\end{enumerate}





\end{document}